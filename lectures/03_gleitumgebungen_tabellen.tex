% !TEX program = xelatex
% !TEX encoding = UTF-8 Unicode
% !TEX spellcheck = de_DE
% 
% © 2016 Moritz Brinkmann, CC-by-sa
% http://latexkurs.github.io

\documentclass[
	vorläufig=true,
	datum=2016-11-11,
	titel={Gleitumgebungen und Tabellen},
	web=false,
	noshortverb=true,
]{../tex/latexkurs-slides}


\usepackage{
	array,
	booktabs,
	caption,
	longtable,
	pifont,
	rotating,
	supertabular,
	tabularx,
	tabulary,
}

%%%%
\newcommand\messdaten{
\toprule
\bf Pendellänge $l$ [m]& \bf Dauer $T$ [s]\\\midrule
4 & 8 \\
2 & 4 \\
1   & 2 \\
.9  & 1.8 \\
0.8 & 1.6 \\
0.7 & 1.4 \\
0.6 & 1.2 \\
0.5 & 1.0 \\
0.4 & 0.8 \\
0.3 & 0.6 \\
0.2 & 0.4 \\
0.1 & 0.2 \\
0.05 & 0.1 \\
0.02 & 0.05 \\
0.01 & 0.02 \\
0.005 & 0.01 \\
0.0025 & 0.005\\
\bottomrule
}
%%%%

\begin{document}


\begin{frame}{Übersicht}
	\tableofcontents
\end{frame}

\section[Gleitumgebungen]{Allgemeine Gleitumgebungen}
\begin{frame}{Was sind Gleitobjekte?}
\begin{itemize}
\item Objekte, die frei im Dokument „gleiten“ können
\item Gleiten vermeidet große Leerräume
\item \TeX\ versucht optimale Positionierung
\item zu beachten:
\begin{itemize}
\item Objekte sollen nicht vor Referenzen auftauchen
\item Objekte sollen nicht die Reihenfolge tauschen
\item Seitenumbruch stark abhängig von Gleitobjekten
\item \emph{optimaler Seitenumbruch ist mit \TeX\ nicht möglich!}
\end{itemize}
\end{itemize}
\end{frame}

\begin{frame}[fragile]{Gleitumgebungen}
Eine Gleitumgebung besteht aus verschiedenen Teilen:
\begin{itemize}
\item Inhalt (Bild, Tabelle, Text, \dots)
\item automatische Bezeichnung: „Tabelle 1:“ (\verb|\caption|)
\item Beschriftung: „Messergebnisse“ (Argument von \verb|\caption{}|)
\item Markierung für Verweise: \verb|\label{fig:messergebnisse}|
\end{itemize}
\end{frame}

\begin{frame}[fragile]{Gleitumgebungen}
\begin{itemize}
\item \LaTeX\ verfügt über verschiedene Gleitumgebungen:
\item \verb|table| für Tabellen
\item \verb|figure| für Abbildungen
\item Paket \pkg{float} ermöglicht Definition eigener Umgebungen
\item für zweispaltigen Satz: \verb|table*|, \verb|figure*| über beide Spalten
\end{itemize}
\end{frame}

\begin{frame}[fragile]{Gleitumgebungen}
\begin{block}{Positionierungsparameter für Gleitumgebungen:} 
\verb|\begin{table}[|\meta{Parameter}\verb|]|
\\
\begin{description}
\item[\texttt{!}] ignoriert Einschränkungen und fährt fort 
\item[\texttt{h}] Objekt genau an dieser Stelle setzen
\item[\texttt{t}] Objekt am Seitenanfang setzen
\item[\texttt{b}] Objekt am Seitenende setzen
\item[\texttt{p}] Objekt in Gleitobjektseite bzw. -spalte setzen
\item[\texttt{H}] „genau hier und sonst nirgends“ – Paket \pkg{float}
\end{description}
\end{block}
\end{frame}

\begin{frame}[fragile]{Gleitumgebungen}
\begin{itemize}
\item Wenn die automatische Positionierung nicht funktioniert:\\%
\verb|\suppressfloats[t,b]|\\
\begin{itemize}
	\item Unterdrückt Positionierung am Kopf oder Fuß der Seite
	\item vermeidet Bilder eines neuen Abschnittes im alten
\end{itemize}
\item nützliche Pakete:
\begin{itemize}
	\item \pkg{placeins}
	\item \pkg{afterpage}
	\item \pkg{endfloat}
\end{itemize}
\end{itemize}
\end{frame}

\begin{frame}[fragile]{table}
\begin{LTXexample}
\begin{table}
  \begin{tabular}{ccc}
    a & b & c
  \end{tabular}
  \caption{Eine sinnlose Tabelle}
  \label{tab:sinnlos}
\end{table}
Im Text kann man auf Tabelle
\ref{tab:sinnlos} verweisen.
\end{LTXexample}
\begin{table}
\begin{tabular}{ccc}
a & b & c
\end{tabular}
\caption{Eine sinnlose Tabelle}
\label{tab:sinnlos}
\end{table}
\end{frame}

\begin{frame}[fragile]{Nichtgleitende Gleitumgebungen}
nichtgleitende Umgebungen als Gleitumgebungen ausgeben:\\
Paket \pkg{caption}

\begin{LTXexample}[pos=b]
Eine kleine Abbildung in einem Text, die eigentlich gar keine ist:
\begin{minipage}[b]{3cm}
  \fbox{ich bin kein Bild}
  \captionof{figure}{test}
\end{minipage}

In der \verb/minipage/ kann jeder beliebige Inhalt stehen \dots
\end{LTXexample}
\end{frame}

\begin{frame}[fragile]{caption}
\pkg{caption} bietet auch vielfältige Einstellungen für Legenden:

\begin{LTXexample}[pos=b]
\captionsetup[figure]{textfont=bf, labelsep=period}
\captionsetup[table]{
  textfont=it, singlelinecheck=false, labelsep=newline, format=plain, justification=justified
}

\begin{figure}
\centering
\fbox{Bild mit \emph{nicht} angepasster Unterschrift – dank Beamer …}
\caption{Unterschrift}
\end{figure}
\end{LTXexample}
\end{frame}

\begin{frame}[fragile]{Drehen von Gleitumgebungen}
\begin{itemize}
\item Paket \pkg{rotating} rotiert den Inhalt um 90° bzw. 270°
\item Umgebungen \verb/sidewaysfigure/, \verb/sidewaystable/
\item nichtgleitend: \verb/sideways/
\end{itemize}
\begin{LTXexample}[width=.4\textwidth]
\centering
\begin{sideways}
[Bild]
\end{sideways}
\captionof{figure}{Nicht gedrehte Beschriftung}
\end{LTXexample}
\end{frame}

\begin{frame}[fragile]{sideways}
\begin{lstlisting}
\begin{sidewaysfigure}
\fbox{Bild}
\caption{Unterschrift}
\end{sidewaysfigure}
\end{lstlisting}
\end{frame}

\part{Tabellen}

\begin{frame}{Tabellen und \LaTeX}
\begin{itemize}
\item[$\pmb-$] Tabellensatz mit \LaTeX\ ist aufwändig!
\item[$\pmb-$] WYSIWYG-Editoren bieten leichtere, da sichtbare Formatierung von Tabellen.
\item[$\pmb+$] Ergebnis sieht in \LaTeX\ meist besser aus.
\item[$\pmb+$] Erscheinungsbild ist frei anpassbar, mit beliebig hohem Aufwand.
\end{itemize}
\end{frame}

\section[tabular]{Standardumgebungen – \texttt{tabular}, \texttt{tabular*}}
\begin{frame}[fragile]{\LaTeX{}s Standardumgebungen}
\begin{itemize}
\item \verb/tabular/, \verb/tabular*/
\item \verb/tabbing/
\item \alert{nicht zu verwechseln mit \texttt{table}!}
\end{itemize}
\end{frame}

\begin{frame}[fragile]{tabular vs. tabbing}
\begin{tabular}[]{rcc}
& \textbf{tabular} & \textbf{tabbing}\\
Eigener Absatz & nein & ja \\
Seitenumbruch & nein & ja  \\
automatische Spaltenbreite & ja & nein\\
Schachtelung & ja & nein
\end{tabular}
\end{frame}

\begin{frame}[fragile,t]{tabbing}
Grundbefehle: \verb|\=, \>|

\begin{LTXexample}[pos=b]
\begin{tabbing}
  erster Eintrag \= zweiter \= dritter \\
  eins \> zwei \> drei\\
  eins \>      \> drei
\end{tabbing}
\end{LTXexample}
\begin{description}
	\item[\texttt{\textbackslash=}] definiert eine neue Tabulatorposition\\
	\item[\texttt{\textbackslash>}] rückt zur nächsten definierten Position vor
\end{description}
\end{frame}

\begin{frame}[fragile,t]{tabbing}
Weitere Befehle: \verb|\kill, \`|

\begin{LTXexample}[pos=b]
\begin{tabbing}
 \hspace{1.5cm} \= \hspace{1cm} \= \qquad \kill
 erster \> zweiter \> dritter \\
 erster Eintrag \> zweiter Eintrag \` dritter Eintrag
\end{tabbing}
\end{LTXexample}
\begin{description}
\item[\texttt{\textbackslash kill}] löscht Inhalt der Zeile, speichert aber die Tabulatoren
\item[\texttt{\textbackslash`}] richtet Text rechtsbündig zum \verb/tabbing/-Rand aus
\end{description}
\end{frame}

\begin{frame}[fragile]{tabular}
\begin{LTXexample}[pos=b,preset={\small}]
\begin{tabular}{l|c||r|p{2cm}@{\ding{53}}c|}
  links & mitte & rechts & vier & fünf\\\hline\hline
  links & mitte &  & eine lange vierte Spalte, die umbrochen wird\\\hline
  & & & &
\end{tabular}
\end{LTXexample}
\end{frame}

\begin{frame}[fragile]{tabular}
\begin{description}
\item[\texttt{l}] linksbündige Spalte
\item[\texttt{c}] zentrierte Spalte
\item[\texttt{r}] rechtbündige Spalte
\item[\texttt{|}] vertikale Linie zwischen Spalten
\item[\texttt{||}] doppelte Linie zwischen Spalten (wird nicht durchgestrichen)
\item[\texttt{p\{\meta{Breite}\}}] Fügt eine \verb|\parbox[t]{|\meta{Breite}\verb|}| ein
\item[\texttt{@\{\meta{Inhalt}\}}] setzt statt Spaltenabstand \verb/Inhalt/
\item[\texttt{\kern-.95ex*\{n\}\{\meta{kürz}\}}] setzt $n$ mal \meta{kürz}, z.\,B. \verb/*{2}{|}/
\end{description}
\end{frame}

\section[booktabs]{Schöne Tabellen – \texttt{booktabs}}

\begin{frame}[fragile]{Fragwürdiges Layout}
\begin{itemize}
\item Paket \pkg{booktabs} (Simon Fear) für hohe Qualität
\item Empfehlungen aus dem Paket:
\end{itemize}
\begin{fancyquote}
\begin{enumerate}
\item \alert{Never, ever use vertical rules.}
\item \alert{Never use double rules.}
\item<2-> Put the units in the column heading (not in the body of the table).
\item<2-> Always precede a decimal point by a digit; thus \verb/0.1/ \emph{not} just \verb/.1/.
\item<2-> Do not use “ditto” signs or any other such convention to repeat a previous
value. In many circumstances a blank will serve just as well. If it won’t,
then repeat the value. \quoted{booktabs-Dokumentation}
\end{enumerate}
\end{fancyquote}
\end{frame}

\begin{frame}[fragile]{ohne booktabs}{Negativbeispiel}
\begin{LTXexample}[width=.45\textwidth,rframe={},preset={\def\ditto{\ --"{}--\ }}]
\begin{tabular}{l||r|r}
  \hline
  Artikel & Zahl & Bezeichnung \\ \hline
  Die & erste & Zeile \\   \cline{2-3}
  Die & zweite & Zeile\\
  Die & dritte & \ditto \\
  Die & vierte & \ditto \\
  \hline
\end{tabular}
\end{LTXexample}
\end{frame}

\begin{frame}[fragile]{mit booktabs}{Positivbeispiel}
\begin{LTXexample}[width=.45\textwidth,rframe={}]
\begin{tabular}{lrr}
  \toprule
  Artikel & Zahl & Bezeichnung \\ \midrule
  Die & erste & Zeile \\   \cmidrule{2-3}
  Die & zweite & Zeile \\
  Die & dritte & Zeile \\
  Die & vierte & Zeile \\
  \bottomrule
\end{tabular}
\end{LTXexample}
\end{frame}

\section[array]{Erweiterungen – \texttt{array}}
\begin{frame}[fragile]{array}
\begin{itemize}
\item Paket \pkg{array} erweitert die Möglichkeiten von \verb/tabular/
\item Änderung von vertikalen Linien, neue Spaltentypen:
\end{itemize}
\begin{description}
\item[\texttt{|}] berücksichtigt die Linienbreite
\item[\texttt{m\{\meta{Breite}\}}] vertikal zentrierte Spalte der angegebenen \meta{Breite}
\item[\texttt{b\{\meta{Breite}\}}] unten ausgerichtete Spalte der angegebenen \meta{Breite} (vgl. p)
\item[\texttt{>\{\meta{Befehl}\}}] fügt \meta{Befehl} direkt vor der nächsten Spalte ein
\item[\texttt{<\{\meta{Befehl}\}}] fügt \meta{Befehl} direkt hinter der letzten Spalte ein
\item[\texttt{!\{\meta{Befehl}\}}] wie \verb/|/, fügt aber \meta{Befehl} ein. Vgl. \verb/@/, aber Abstand korrigiert
\end{description}
\end{frame}

\begin{frame}[fragile]{array}
\begin{LTXexample}[pos=b]
\begin{tabular*}{6cm}{|p{1cm}p{3cm}p{1cm}|}
links & mittlerer Text mit eingebautem Umbruch & rechts
\end{tabular*}
\end{LTXexample}
\end{frame}
\begin{frame}[fragile]{array}
\begin{LTXexample}[pos=b]
\begin{tabular*}{6cm}{|m{1cm}m{3cm}m{1cm}|}
links & mittlerer Text mit eingebautem Umbruch & rechts
\end{tabular*}
\end{LTXexample}
\end{frame}
\begin{frame}[fragile]{array}
\begin{LTXexample}[pos=b]
\begin{tabular*}{6cm}{|b{1cm}b{3cm}b{1cm}|}
links & mittlerer Text mit eingebautem Umbruch & rechts
\end{tabular*}
\end{LTXexample}
\end{frame}
\begin{frame}[fragile]{array}
\begin{LTXexample}[pos=b]
\begin{tabular}{>{\bfseries}l|>{\color{red}}r}
links & rechts\\
links & rechts
\end{tabular}
\end{LTXexample}
\end{frame}

\section[tabularx, tabulary]{Automatische Breite – \texttt{tabularx}, \texttt{tabulary}}
\begin{frame}[fragile]{tabular*}
\begin{itemize}
\item \verb/tabular*/ ändert \emph{Abstand} der Spalten
\item \verb/tabularx/ verteilt \emph{Breite} der Spalten \emph{gleichmäßig}
\item \verb/tabulary/ verteilet \emph{Breite} der Spalten \emph{am Inhalt orientiert}
\end{itemize}
\end{frame}

\begin{frame}[fragile]{automatische Breiten}
\begin{LTXexample}[width=.4\textwidth]
\begin{tabular*}{4cm}{|l|!{\extracolsep\fill}>{(}l<{)}|r|}
a a & b b & c c
\end{tabular*}
\\ \\
\begin{tabular}{|l|!{\extracolsep\fill}l|r|}
a a & b b & c c
\end{tabular}
\\ \\
\begin{tabularx}{4cm}{|l|>{(}X<{)}|r|}
a a & b b & c c
\end{tabularx}
\end{LTXexample}
\end{frame}

\begin{frame}[fragile]{tabularx}
Automatische Berechnung der Spaltenbreite:
\begin{LTXexample}
\begin{tabularx}{\linewidth}{lX|X|r}
linke Spalte & Eine längere Spalte& kurz & rechts
\end{tabularx}
\end{LTXexample}
\end{frame}

\begin{frame}[fragile]{tabulary}
\begin{LTXexample}
\begin{tabulary}{4cm}{|L|L|L|}
a & b b b b b b b b b & c c c c c c c c c c c c c c c c c 
\end{tabulary}
\end{LTXexample}
\begin{LTXexample}
\begin{tabular}{|l|l|l|}
a & b b b b b b b b b & c c c c c c c c c c c c c c c c c 
\end{tabular}
\end{LTXexample}
\begin{LTXexample}
\begin{tabular*}{4cm}{|l|l|l|}
a & b b b b b b b b b & c c c c c c c c c c c c c c c c c 
\end{tabular*}
\end{LTXexample}
\end{frame}

\begin{frame}[fragile]{tabulary}
Mögliche Spaltentypen:\\[1em]
\begin{description}
\item[\texttt{L}] linksbündig
\item[\texttt{R}] rechtsbündig
\item[\texttt{C}] zentriert
\item[\texttt{J}] Blocksatz
\end{description}
\vfill
\begin{itemize}
\item Alle Spalten verhalten sich wie \verb/p/-Spalten.
\item Breite der Spalten ist \emph{nicht} vorher festgelegt.
\end{itemize}
\end{frame}

\section[supertabular, longtable]{Mehrseitige Tabellen – \texttt{supertabular}, \texttt{longtable}}
\begin{frame}[fragile]{lange Tabellen}
Lösung: \pkg{supertabular} oder \pkg{longtable}\\[1em]
\begin{description}
\item[\texttt{supertabular}] mehrseitige Tabelle, Breite variabel
\item[\texttt{supertabular*}] festgesetzte Breite
\item[\texttt{mpsupertabular}] setzt Tabelle in \verb/minipage/
\item[\texttt{mpsupertabular*}] \verb/minipage/ mit fester Breite
\end{description}
\end{frame}

\begin{frame}[fragile]{supertabular}
\begin{supertabular}{cc}
\messdaten
\end{supertabular}
\end{frame}

\begin{supertabular}{cc}
\messdaten
\end{supertabular}

\begin{frame}[fragile]{supertabular}
Wichtige Einstellungsmöglichkeiten:
\begin{lstlisting}
\tablehead{links & rechts \\\hline}
\tablefirsthead{\bf links & \bf rechts \\}
\tabletail{\small \textit{Fortsetzung auf der nächsten Seite} & \\}
\tablelasttail{Ende der Messdaten}
\end{lstlisting}
\tablehead{links & rechts \\\hline}
\tablefirsthead{\bf links & \bf rechts \\}
\tabletail{\small \textit{Fortsetzung auf der nächsten Seite} & \\}
\tablelasttail{Ende der Messdaten}
\end{frame}

\begin{supertabular}{cc}
\messdaten
\end{supertabular}

\begin{frame}[fragile]{longtable}
Paket \verb|longtable| bietet Umgebung \verb|longtable|:
\begin{itemize}
\item feste Breite der Spalten auf allen Seiten
\item \verb/head/, \verb/firsthead/ etc. werden \emph{innerhalb} der Tabelle festgelegt
\item verwendet die .aux-Datei (auf Schreibrechte achten!)
\end{itemize} 
\end{frame}

\begin{frame}[fragile]{longtable}
\begin{LTXexample}[pos=b]
\begin{longtable}{cc}
\textbf{Messdaten}\\
\endfirsthead
links & rechts\\
\endhead
\small \textit{Weiter auf der nächsten Seite}
\endfoot
Ende der Tabelle.
\endlastfoot
\messdaten
\end{longtable}
\end{LTXexample}
\end{frame}

\begin{longtable}{cc}
\textbf{Messdaten}\\
\endfirsthead
links & rechts\\
\endhead
\small \textit{Weiter auf der nächsten Seite}
\endfoot
Ende der Tabelle.
\endlastfoot
\messdaten
\end{longtable}

\iffalse
\begin{frame}[fragile]{supertabularx, longtablex}
Für Satz mehrseitiger Tabellen mit automatischer Breitenanpassung:
\only<1>{\texttt{supertabularx} bzw. \texttt{longtablex}}
\only<2->{\sout{\texttt{supertabularx} bzw. \texttt{longtablex}}}
\pause\pause
\begin{itemize}
\item Paket \verb/ltxtable/ bietet grundlegende Unterstützung
\item Kombination von \verb/longtable/ und \verb/tabularx/
\item Tabelle (\texttt{tabularx}) selbst steht in externer Datei
\item Nutzer muss diese selbst anlegen, schreiben und verwalten
\item Einbinden mittels \texttt{\textbackslash LTXtable\{\meta{width}\}\{\meta{file}\}}
\item am besten mittels \verb/filecontents/ (Umgebung, Paket)
\end{itemize}
\end{frame}
\fi

\begin{frame}[fragile]{Zellen über mehrere Spalten/Zeilen}
Mit \verb|\multicolumn{|\meta{Spalten}\verb|}{|\meta{Ausrichtung}\verb|}{|\meta{Inhalt}\verb|}| kann eine Zelle mehrere Spalten überdecken.
\begin{lstlisting}
\multicolumn{2}{c}{Zelle über zwei Spalten (zentr.)}
\end{lstlisting}
\pause\vfill
Paket \pkg{multirow} bietet Unterstützung für Zellen über mehrere Zeilen.
\verb|\multirow{|\meta{Zeilen}\verb|}{|\meta{Breite}\verb|}{|\meta{Inhalt}\verb|}|
\begin{lstlisting}
\multirow{3}{*}{Zelle über drei Zeilen}
\end{lstlisting}
\end{frame}

\section[weiteres]{weitere nützliche Pakete}
\begin{frame}[fragile]{weitere nützliche Pakete}
\begin{description}
\item[\pkg{colortbl}] farbige Linien
\item[\pkg{hhline}] vielfältige Linien (horizontal, vertikal \dots)
\item[\pkg{arydshln}] gestrichelte Linien 
\item[\pkg{tabls}] Zeilenabstände einstellen (inkompatipel zu \verb|array|!)
\item[\pkg{ltxtable}] mehrseitige Tabellen mit automatischer Breitenanpassung
%\item[\pkg{multirow}] vertikale Ausrichtung 
\item[\pkg{dcolumn}] Ausrichtung am Dezimalpunkt % das kann auch siunitx → siehe nächste Vorlesung
\item[\pkg{threeparttable}] Fußnoten an Tabellen
\end{description}
\end{frame}


\begin{frame}{Weiterführende Literatur}
	\begin{thebibliography}{9}
		\bibitem{booktabs} \textsc{Simon Fear}:
			\newblock{\href{http://mirrors.ctan.org/macros/latex/contrib/booktabs/booktabs.pdf}{„Publication quality tables in \LaTeX“}}, 
			\newblock{\href{http://mirrors.ctan.org/macros/latex/contrib/booktabs/booktabs.pdf}{\texttt{texdoc booktabs}}}.
		\setbeamertemplatebibitembook
		\bibitem{voss} \textsc{Herbert Voss}:
			\newblock{„Tabellen mit LateX“},
			\newblock{Lehmanns Media, 2010}.
	\end{thebibliography}
\end{frame}
\end{document}