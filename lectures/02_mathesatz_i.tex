% !TEX program = xelatex
% !TEX encoding = UTF-8 Unicode
% !TEX spellcheck = de_DE
% 
% © 2016 Moritz Brinkmann, CC-by-sa
% http://latexkurs.github.io

\documentclass[
	vorläufig=false,
	datum=2016-11-04,
	titel={Mathematiksatz I},
	web=false,
]{../tex/latexkurs-slides}

\usepackage{
	mathtools,
	ulem,
}
\normalem


\usetikzlibrary{calc,intersections,through, positioning}

\tikzset{onslide/.code args={<#1>#2}{%
  \only<#1>{\pgfkeysalso{#2}} % \pgfkeysalso doesn't change the path
}}

\begin{document}

\begin{frame}{Übersicht}
	\tableofcontents
\end{frame}

\section{Eigene Befehle}
\frame{\centering\alert{Teil I}\\\huge Eigene Befehle definieren}
\begin{frame}[fragile]{Eigene Befehle}
\overleaf{tex0201}
\begin{itemize}
\item |\newcommand{\wasser}{H$_2$O}| ⇒ H$_2$O
\item Ermöglicht Abkürzungen im Text, die häufig vorkommen\pause
\item Änderung: |\renewcommand{\wasser}{H\kern-.1em$_2$\kern-.1em O}|:\\ H\kern-.1em$_2$\kern-.1em O
\end{itemize} 
\end{frame}

\begin{frame}[fragile,t]{Leerzeichen in \TeX}
\overleaf{tex0201}
\TeX\ „frisst“ gerne Leerzeichen – vor allem nach Befehlen:\\
|\wasser ist nass| ⇒ H$_2$Oist nass.

\pause
\begin{itemize}
\item<+-> \TeX\ liest Befehle vom |\| bis zum ersten nicht-Buchstaben%
\\ (Zahl, Klammer, Leerzeichen, Punkt, \dots)
\\ \verb*|\LaTeX   ist  manchmal   umständlich|%
\item<+-> Befehle im Text immer mit |\| oder |{}| beenden:
\item<+-> \verb*|\LaTeX\ ist manchmal umständlich.|
\end{itemize}
\begin{block}{}\hspace{7em}{%
\only<2-3>{\LaTeX ist manchmal umständlich}%
\only<4>{\LaTeX\ ist manchmal umständlich}}
\end{block}
\end{frame}

\begin{frame}[fragile]{Befehle mit Argumenten}
\begin{lstlisting}
\newcommand\molekuel[3][H]{Das Molekül #1$_#2$#3}
\end{lstlisting}
\begin{itemize}
\item Argumente werden mit |[|\meta{Anzahl}|]| definiert
\item Optionales Argument in eckigen Klammern
\item Zugriff in der Definition möglich mit |#1|
\item In der Verwendung meist mit geschweiften Klammern |{Co}|
\end{itemize}
\newcommand\molekuel[3][H]{Das Molekül #1$_#2$#3}
|\molekuel{2}{O}| ⇒ \molekuel{2}{O}\\
|\molekuel[Co]{7}{O}| ⇒ \molekuel[Co]{7}{O}\\
\end{frame}

\frame{\centering\alert{Teil II}\\\huge Mathe}

\section{Mathe: inline vs. display}
\subsection{Inlinemode}
\begin{frame}[fragile]{Inlinemode}
\begin{itemize}
	\item Formeln, die direkt im Fließtext vorkommen
	\item kurze Formeln, Nennung von Variablen
	\item Elemente gehen nicht über die Zeilenhöhe hinaus
	\item Grenzen werden \emph{neben} Integrale, Summen und Produkte gesetzt
\end{itemize}
\vfill
\rmfamily
\begin{LTXexample}
Seien $m$ und $n$ natürliche Zahlen mit $n=5 m$.
\end{LTXexample}
\end{frame}

\begin{frame}{Inline- vs. Display-Formeln}
\rmfamily
\textbf{Inline-Mathe:} $E=mc^2$ kennt jedes Kind, aber kaum jemand kann wirklich mehr damit anfangen als mit $\int^\infty_{-\infty}\sum_{n = 1}^5 dx$, wobei diese Formel nun mal gar keinen Sinn ergibt, aber zeigt, wie Grenzen im \TeX-Mathesatz aussehen.
\textbf{Inline-Mathe mit Displaystyle:} $E=mc^2$ kennt jedes Kind, aber kaum jemand kann wirklich mehr damit anfangen als mit $\displaystyle \int^\infty_{-\infty}\sum_{n = 1}^5 dx$, wobei diese Formel nun mal gar keinen Sinn ergibt, aber zeigt, wie Grenzen im \TeX-Mathesatz aussehen.
\textbf{Display-Mathe:} $E=mc^2$ kennt jedes Kind, aber kaum jemand kann wirklich mehr damit anfangen als mit \[\int^\infty_{-\infty}\sum_{n = 1}^5 dx,\] wobei diese zweite Formel nun mal gar keinen Sinn ergibt, aber zeigt, wie Grenzen im \TeX-Mathesatz aussehen.
\end{frame}

\begin{frame}[fragile]{Inlinemode}
Der Inlinemode ist über drei Wege zu erreichen:
\begin{itemize}
\item |\(|\meta{Formel}|\)|
\item |\begin{math}|\meta{Formel}|\end{math}|
\item |$|\meta{Formel}|$|
\end{itemize}
\overleaf{tex0201}
\pause
|$ $| ist meist die beste Variante
\end{frame}

\begin{frame}[fragile]{Umbrüche}
Formeln können von \TeX{} umgebrochen werden:
\begin{itemize}
\item an Relationen \hfill |=|, |<|, |>|, etc.
\item an binären Operatoren \hfill |+|, |-|, etc.
\item Umbruch kann durch Gruppierung vermieden werden. \hfill |{}|
\end{itemize}
\begin{LTXexample}
Ein Text bis zum Zeilenende 
$a + b + c$ \\
Ein Text bis zum Zeilenende 
${a + b + c}$
\end{LTXexample}
\end{frame}

\subsection{Displaymode}
\begin{frame}{Displaymode}
	\begin{itemize}
		\item Auszeichnung wichtiger Formeln
		\item Darstelling langer Rechnungen
		\item komplexe Formeln
		\item mehrfach indizierte Größen
		\item geschachtelte Brüche
		\item …
	\end{itemize}
\end{frame}

\begin{frame}[fragile]{Displaymode}
\emph{klassische} Display-Formeln sind über drei Wege zu erreichen:
	\begin{itemize}
		\item |\begin{displaymath}|\meta{Formel}|\end{displaymath}|\\
		abgesetzte Formel ohne Nummerierung
		\item |\[|\meta{Formel}|\]|\\
		Abkürzung für |displaymath|
		\item |\begin{equation}|\meta{Formel}|\end{equation}|\\
		abgesetzte Formel mit Nummerierung
		\item<2> \sout{\texttt{\$\$}\meta{Formel}\texttt{\$\$}}\\\pause
		\TeX-Syntax führt in \LaTeX{} zu unerwarteten und unerwünschten Ergebnissen\\\alert{⇒ unbedingt vermeiden!}
	\end{itemize}
\end{frame}

\begin{frame}[fragile]{Display in Inline und umgekehrt}
	\begin{itemize}
		\item Dislaystyle kann mit |\displaystyle| im Inline-Modus aufgerufen werden.
	\end{itemize}
\begin{LTXexample}[pos=b]
Hier kommt ein großer Bruch, der
$\frac{a}{b} < \displaystyle \frac{a}{b}$
viel zu groß für den normalen Fließtext ist.
\end{LTXexample}
	\begin{itemize}
		\item Inlinestyle kann mit |\textstyle| im Display-Modus aufgerufen werden.
	\end{itemize}
\begin{LTXexample}[pos=b]
\[\frac 12 > \textstyle \frac 12 \]
\end{LTXexample}
\end{frame}

\begin{frame}[fragile]{Option fleqn}
	\overleaf{tex0201}
	\begin{itemize}
		\item Formeln sehen oft zentriert nicht gut aus und wirken zerfleddert
		\item linksbündige Ausrichtung ggf. besser
		\item[⇒] |fleqn| als Dokumentenoption 
	\end{itemize}
\begin{lstlisting}
\documentclass[fleqn]{scrartcl}
\end{lstlisting}
\end{frame}

\section{\texttt{amsmath}}
\begin{frame}[fragile]{Mehrzeilige Formeln}
	Eine Reihe von untereinander ausgerichteten, zueinander angeordneten Gleichungen wird z.\,B. verwendet für:
	\begin{itemize}
		\item Herleitungen
		\item Übersichten
		\item Vergleich von Formeln
	\end{itemize}
	\pause
	\sout{\TeX-Standardumgebung: \texttt{eqnarray}} \pause \alert{unschön}\\
	\alert{besser:} |align|-Umgebung aus dem |amsmath|-Paket.
	\pause
\begin{LTXexample}
\begin{align}
a &= b, &
c &= d,\\
abc &= d \\
&= r
\end{align}
\end{LTXexample}
	ohne Nummerierung: |{align*}|
\end{frame}

\begin{frame}[fragile]{\hologo{AmS}math}
	\begin{itemize}
		\item Paket von der American Mathematical Society (\hologo{AmS})
		\item besteht aus mehreren Paketen, u.\,a.:\\%
		|amsmath|, |amssymb|, |amsfonts|%
		\item bietet umfangreiche Erweiterungen des Mathesatzes:
		\item vielfältige Umgebungen und Anpassungen
		\item neue oder verbesserte Definitionen von Befehlen
		\item Korrekturen von Abständen
		\item \only<1>{\dots}
		\pause wird mit Fehlerkorrekturen, etc. ergänzt durch |mathtools| 
		\pause
		\item[⇒] kann im Prinzip \emph{immer} geladen werden, wenn man was mit Mathe macht.
	\end{itemize}
\begin{lstlisting}
\usepackage{amsmath, mathtools}
\end{lstlisting}\pdfmarginpar{Genaugenommen reicht es aus, mathtools zu laden. Es läd amsmath automatisch.}
\end{frame}

\section{Grundbefehle}
\subsection{Abstände}
\begin{frame}[fragile]{Abstände}
	\begin{itemize}
		\item \TeX\ bzw. \LaTeX\ bzw. geladene Pakete kontrollieren Abstände
		\item Unterschiede zwischen Variablen, Operatoren, Relationen etc.
		\item Festgelegt durch die |\mathcode|s der Zeichen
		\item Änderbar mit |\kern|, |\|, |\,| etc.
		\item \alert{niemals} Konstrukte wie |\ \ \ \ | verwenden!
		\item Besser: |\quad|, |\qquad|, |\hspace{1em}|
	\end{itemize}
\end{frame}

\subsection{Größe von Formeln}
\begin{frame}[fragile]{Größenänderungen}
	\begin{itemize}
		\item Standardbefehle wie |\small|,  |\tiny|, |\Huge| haben in Formeln keine Wirkung
		\item Aber Formeln passen sich der Umgebung an
	\end{itemize} 
	\pause
\begin{LTXexample}[pos=b]
\small \[ 
  x_{n+1} = x_n - \frac{f(x_n)}{f^\prime(x_n)} 
\]
\huge \[
  x_{n+1} = x_n - \frac{f(x_n)}{f^\prime(x_n)}
\]
\end{LTXexample}
\end{frame}

\section{Variablen}
\begin{frame}[fragile]{Variablen und Zahlen}
	\begin{itemize}
		\item Variablen werden kursiv gesetzt: |$a$|: $a$
		\item Schriftart abhängig von der Dokumentenklasse!\\%
		(Groteske, Serifen etc.)
		\item Ziffern werden automatisch korrekt gesetzt: $12.2$ statt 12.2
	\end{itemize}
\end{frame}

\begin{frame}[fragile]{Dezimaltrennzeichen}
im amerikanischen Satz:
\begin{LTXexample}[preset=\Large]
$1,234.567$
\end{LTXexample}\pause
im deutschen Satz:
\begin{LTXexample}[preset=\Large]
$1.234,567$
\end{LTXexample}
\alert{⇒ falsche Spationierung!}
\end{frame}

\begin{frame}[fragile]{Dezimaltrennzeichen}
\begin{block}{Einmalige Anpassung:}
	|$1\mathpunct{.}234\mathpunct{.}567{,}89$|\\
	$1\mathpunct{.}234\mathpunct{.}567{,}89$ (angepasst)\\
	$1.234.567,89$ \normalsize (nicht angepasst)\\
\end{block}
%\pause
%\begin{block}{Korrektur des Dezimaltrennzeichens}
%|\DeclareMathSymbol{,}{\mathpunct}{letters}{"3B}|\\
%|\DeclareMathSymbol{.}{\mathord}{letters}{"3A}|
%\end{block}
\pause
\begin{block}{Automatische Anpassung}
Paket |icomma| passt Dezimaltrennzeichen automatisch dokumentenweit an.\\
Andere Möglichkeit: Paket |siunitx| → siehe Vorlesung Mathesatz II
\end{block}
\end{frame}

\begin{frame}[fragile]{Hoch- und Tiefstellung}
	\begin{itemize}
		\item Zeichen mit besonderer Bedeutung: |^| und |_|
		\item Hochstellung: |a^b|\hfill $a^b$
		\item Tiefstellung: |a_b|\hfill $a_b$
		\item Gruppierungen sind möglich: |a^{bc}|, |a_{bc}|\hfill $a_{bc}$
		\item Kombination ist möglich: |a_b^c|\hfill $a_b^c$
		\item Ohne vorhergehendes Zeichen: |^{235}U|\hfill $^{235}\mathrm U$
		\item Schachtelung nur mit Gruppierung:\\%
		|a_{b_{c_{d_{e_{f^g}}}}}^{h^{i^{j_k}}}| \hfill \Large $a_{b_{c_{d_{e_{f^g}}}}}^{h^{i^{j_k}}}$\normalsize
		\item[] |a_b_c| produziert Fehler!
	\end{itemize}
\end{frame}

\subsection{Operatoren}
\begin{frame}[fragile]{Operatoren}
Operatorennamen werden aufrecht gesetzt und sind vordefiniert
	\begin{itemize}
		\item richtig: $\sin(x)$\quad falsch: $sin(x)$
	\end{itemize}
\begin{LTXexample}[pos=b]
$\sin(x) \cos(y) \tan(2\pi) \lim \arctan$
\end{LTXexample}
	\pause
	\begin{itemize}
		\item Paket amsopn bietet viele Vordefinitionen:
	\end{itemize}
\begin{lstlisting}
\arccos \arcsin \arg \cos \cot \coth \deg \det
\exp \gcd \inf \injlim \lg \lim \limsup \ln
\max \min \projlim \sec \sinh \sup \tanh
\end{lstlisting}
\end{frame}

\begin{frame}[fragile]{Operatoren}
Sollten die vorgegebenen Definitionen nicht genügen:
\begin{lstlisting}
\usepackage{amsopn}
\DeclareMathOperator{\Res}{Res}
\end{lstlisting}
in der Präambel.
\end{frame}

\begin{frame}[fragile]{Klammern}
Klammerung von großen Ausdrücken kann Probleme bereiten:
\begin{LTXexample}
\[ (
  \frac{\int^a x dx}{\sum_{n=1} x}
) \]
\end{LTXexample}
Besser:
\begin{LTXexample}
\[ \left(
  \frac{\int^a x dx}{\sum_{n=1} x}
\right) \]
\end{LTXexample}
\end{frame}


\begin{frame}[fragile]{Klammern}
\begin{itemize}
\item |\left| und |\right| vor allem, was dehnbar ist
\item |\left( \right]| funktioniert auch
\item |\left. \right)| liefert angepasste rechte Klammer
\item Hoch- und Tiefstellung werden angepasst:
\end{itemize}
\begin{LTXexample}[pos=b]
\begin{displaymath}
  \left. \int_a^b f(x) \mathrm dx \right\vert_a^b
  \qquad
  \left\{ \int_a^b f(x) \mathrm dx \right]
\end{displaymath}
\end{LTXexample}
\end{frame}

\begin{frame}[fragile]{Grenzen}
\begin{itemize}
\item Grenzen per |\limits| angeben
\item Mehrzeilige Grenzen mit |\atop|
\end{itemize}
\begin{LTXexample}
\[
  \int_a^b
  \int\limits_a^b
  \sum_{n=1}^\infty
  \prod_{n = 1 \atop m = 2}
\]
\end{LTXexample}
\end{frame}

\begin{frame}[fragile]{Sonderzeichen}
\begin{itemize}
\item Viele Zeichen sind über ihren Namen ereichbar,
\item genauso Griechische Groß- und Kleinbuchstaben
\end{itemize}
\begin{LTXexample}[preset=\vspace{-1em}]
\begin{align*}
  \nabla \square \\ 
  \partial \infty \\
  \pm \mp \\
  \alpha \beta \gamma \\
  \rho \varrho \\
  \kappa \varkappa \\
  \epsilon \varepsilon \\
  \theta \vartheta \\
   A B \Gamma 
\end{align*}
\end{LTXexample}
\pause
Wenn man ein Symbol sucht:\\
\texttt{texdoc \href{http://mirrors.ctan.org/info/symbols/math/maths-symbols.pdf}{maths-symbols} \href{http://mirrors.ctan.org/info/symbols/comprehensive/symbols-a4.pdf}{symbols-a4}}
oder \alert{\href{http://detexify.kirelabs.org/classify.html}{Detexify}}
\end{frame}

\begin{frame}[fragile]{Wurzeln}
\begin{LTXexample}[preset=\Large]
\[
  \sqrt{a_{n_{m_p}}}
  \quad
  \sqrt[3]{a}\quad
\]
\end{LTXexample}
\pause
\begin{itemize}
\item zu tiefe Unterlängen sind unschön \item[⇒] |\smash[|\meta{t, b}|]{|\meta{Formel}|}|
\end{itemize}
\begin{LTXexample}[preset=\Large]
\[
  \sqrt{a_{n_{m_p}}}
  \quad
  \sqrt{
    \smash[b]{
      a_{n_{m_p}}
    }
  }
\]
\end{LTXexample}
\end{frame}

\section{Vektoren, Matrizen, Tensoren}
\begin{frame}[fragile,t]{Vektoren}
Vektoren sind vielfältig darstellbar:
\begin{itemize}
\item Fettgedruckt  mit |\boldsymbol| oder |\mathbf|
\item „falscher“ Fettdruck: |\pmb|
\item Mit Pfeil drüber als |\vec|\\[-1.28em]
\only<2>{
\hspace*{-.76ex}\vspace*{-1.65ex}

$\left.
\begin{matrix}
	\text{\quad}\\%[-.5ex]
	\text{\hspace{12em}}
\end{matrix}
\right\rbrace\text{\parbox{9em}{\raggedright Typografisch unschön, nur für Handschriften}}$
\vspace*{-2.9ex}
}
\item Unterstrichen mit |\underbar|
\end{itemize}
\vfill
\begin{LTXexample}[pos=b, preset=\centering]
$ \boldsymbol a\ \mathbf a $ \\
$ \pmb a\ a $ \\
$ \vec a\ \underbar a $
\end{LTXexample}
\end{frame}

\begin{frame}[fragile]{Matrizen}
\begin{LTXexample}
\[
  \begin{matrix}
    a_{11} & a_{12}\\
    a_{21} & a_{22}
  \end{matrix}
\]
\end{LTXexample}
\pause
\begin{LTXexample}
\[
  \left(
    \begin{matrix}
      a_{11} & a_{12}\\
      a_{21} & a_{22}
    \end{matrix}
  \right)
\]
\end{LTXexample}
\end{frame}


\begin{frame}[fragile]{Matrizen}
\AmS math definiert weitere Matrixumgebungen:\\[2em]
\begin{minipage}{3cm}
\[\begin{pmatrix}
a & b \\ c & d
\end{pmatrix}\]
\centering pmatrix
\end{minipage}
\begin{minipage}{3cm}
\[\begin{Vmatrix}
a & b \\ c & d
\end{Vmatrix}\]
\centering Vmatrix
\end{minipage}
\begin{minipage}{3cm}
\[\begin{vmatrix}
a & b \\ c & d
\end{vmatrix}\]
\centering vmatrix
\end{minipage}
\\[2em]
\begin{minipage}{3cm}
\[\begin{Bmatrix}
a & b \\ c & d
\end{Bmatrix}\]
\centering Bmatrix
\end{minipage}
\begin{minipage}{3cm}
\[\begin{bmatrix}
a & b \\ c & d
\end{bmatrix}\]
\centering bmatrix
\end{minipage}
\begin{minipage}{3cm}
\[\begin{smallmatrix}
a & b \\ c & d
\end{smallmatrix}\]\\
\centering smallmatrix
\end{minipage}
\end{frame}

\nocite{mathmode, dante:mathe, amsmath}
\begin{frame}[allowframebreaks]{Weiterführende Literatur}
\printbibliography
\end{frame}

% Was noch fehlt: Definition von Kommandos
% Normen
% Skalarprodukte

\end{document}