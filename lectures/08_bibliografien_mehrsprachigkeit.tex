% !TEX program = xelatex
% !TEX encoding = UTF-8 Unicode
% !TEX spellcheck = de_DE
% 
% © 2016 Moritz Brinkmann, CC-by-sa
% http://latexkurs.github.io

\documentclass[
	vorläufig=false,
	datum=2017-01-13,
	titel={Bibliografien und mehrsprachiger Satz},
	web=false,
%	noshortverb=true,
]{../tex/latexkurs-slides}

\setmonofont[Scale=0.9]{Anonymous Pro}

\usepackage{
	booktabs,
	biblatex,
}

\begin{document}

\begin{frame}{Übersicht}
	\tableofcontents
\end{frame}

\section{Bibliografie}
\begin{frame}[fragile]{Bibliografie}
\begin{itemize}
	\item Bibliografie enthält Liste verwendeter Quellen und ggf. weiterführende Literatur.
	\item je nach Fachbereich unterschiedliche Zitierstile
	\item (grobes) Aussehen der Bibliografie wird von Dokumentenklasse bestimmt.
	\item bestimmte Syntax zum Setzen der Bibliografie:
	\begin{itemize}
		\item Umbegung |\begin{thebibliography}{|\meta{Anzahl}|}|
		\item Aufzählung der Werke mittels |\bibitem{|\meta{Key}|}| \meta{Text}
		\item Zitieren eines Werks mit |\cite{|\meta{Key(s)}|}| oder |\cite[|\meta{Seite}|]{|\meta{Key}|}|
	\end{itemize}
\end{itemize}
\end{frame}


\begin{frame}[fragile,t]{Bibliografie}
\begin{lstlisting}
\begin{thebibliography}{9}
  \bibitem{frankfurt05} Harry G. Frankfurt:
    \textit{On Bullshit}, Princeton University Press,
    Princeton, New Jersey, 2005.
\end{thebibliography}
\end{lstlisting}
\vfill
\pause
\begin{itemize}
	\item manuelles Erstellen (und Sortieren) der Bibliografie ist sehr umständlich
	\item Einträge nicht sinnvoll wiederverwendbar
	\pause
	\item[⇒] Programm \BibTeX\ übernimmt Sortierung und Verwaltung der Einträge
\end{itemize}
\end{frame}

\subsection{\BibTeX}
\begin{frame}[fragile]{\BibTeX-Idee}
	\begin{itemize}
		\item Einträge liegen als Textdatei (|.bib|) in vorgegbener Syntax vor
		\item Referenz im Dokument mit |\cite{mittelbach2004}|\pdfmarginpar{Wenn man will, dass nicht zitierte Referenzen in der Bibliografie auftauchen kann man diese mit nocite{} hinzufügen. nocite{*} fügt alle Items in der .bib-Datei ein.}
		\item Programm \BibTeX\ fügt referenzierte Quelle automatisch in Bibliografie ein
		\item Aussehen der Referenz und Bibliografieeinträge vielfältig einstellbar
		\item Zugriff auf große Menge an verfügbaren Referenzen
	\end{itemize}
\end{frame}

\begin{frame}{\BibTeX}
	\begin{itemize}
		\item Verwendung unintuitiv
		\item graphische Oberflächen erleichtern das Leben
		\item z.\,B. JabRef, Citavi, etc.
		\item direkte online-Suche z.\,B. bei \url{https://scholar.google.de}
	\end{itemize}
\end{frame}

\begin{frame}[fragile]{Pakete}
	Gestaltung der Bibliographie mittels Paketen:
	\begin{description}
		\item[\pkg{natbib}] numerische [13] oder Author-Year (Tolkien, 1954) Zitierstile (und Varianten)
		\item[\pkg{jurabib}] Zitierstile für Jura (und Humanities)
		\item[\pkg{inlinebib}] für Zitate in Fußnoten
		\item …\pause
		\item[\pkg{biblatex}] vielfältigste Gestaltungsmöglichkeiten (empfohlen)
\end{description}\vfill
Programm \pkg{biber} als Nachfolger von \BibTeX\ für \pkg{biblatex}
\end{frame}

\subsection{natbib}
\begin{frame}[fragile,shrink]{Erstellung der Bibliografie}{|natbib|}
\begin{block}{im Dokument}
\begin{lstlisting}
\usepackage[optionen]{natbib}
\begin{document}
  \bibliographystyle{plainnat} % oder andere ...
  Text ... \cite{Tolkien54} ... text.
  \bibliography{Bib-Datei}
\end{document}
\end{lstlisting}
\end{block}
\begin{block}{in der .bib-Datei}
\begin{lstlisting}
@book{Tolkien54,
  author    ={Tolkien, John R. R.},
  title     ={The Lord of the Rings},
  publisher ={Allen & Unwin},
  address   ={London},
  year      ={1954},
}
\end{lstlisting}
\end{block}
\end{frame}

\begin{frame}[fragile]{Elemente der |.bib|-Datei}
Unterschiedliche Bib-Items für unterschiedliche Dokumenttypen:
\begin{columns}
\begin{column}{.3\textwidth}
	\begin{itemize}
		\item |@article|
		\item |@book|
		\item |@mvbook|
		\item |@inbook|
		\item |@suppbook|
	\end{itemize}
\end{column}
\begin{column}{.3\textwidth}
	\begin{itemize}
		\item |@collection|
		\item |@manual|
		\item |@online|
		\item |@patent|
		\item |@periodical|
	\end{itemize}
\end{column}
\begin{column}{.4\textwidth}
	\begin{itemize}
		\item |@proceedings|
		\item |@thesis|
		\item |@unpublished|
		\item …
		\item[]
\end{itemize}
\end{column}
\end{columns}
\vfill
Jedes Item hat verschiedene mandatorische und optionale Felder.
\vfill
\begin{block}{Syntax eines Eintrags}
|@|\meta{Item-Typ}|{|\meta{Ref-Key}|,|\\
|    |\meta{Feld}|    = {|\meta{Wert}|},|\\
|    |\meta{Feld}|    = {|\meta{Wert}|},|\\
|}|
\end{block}
\end{frame}

\begin{frame}[fragile]{Setzten der Bibliografie}{\BibTeX}
\begin{block}{in der Konsole}
|$ xelatex dokument.tex|\\
|$ bibtex dokument.aux|\\
|$ xelatex dokument.tex|
\end{block}
\end{frame}

\begin{frame}[fragile, shrink]{Zitierbefehle}{|natbib|}
\small\hspace*{-1cm}
\begin{tabular}{lp{3.5cm}l}
\bf Befehl & \bf Ausgabe Author-Year & \bf Ausbgabe num. \\ \midrule
|\citet{jon90}|	&	Jones et al. (1990)	&	Jones et al. [21]	\\
|\citet[chap.~2]{jon90}|	&	Jones et al. (1990, chap. 2)	&	Jones et al. [21, chap. 2]	\\\\
|\citep{jon90}|	&	(Jones et al., 1990)	&	[21]	\\
|\citep[chap.~2]{jon90}|	&	(Jones et al., 1990, chap. 2)	&	[21, chap. 2]	\\
|\citep[see][]{jon90}|	&	(see Jones et al., 1990)	&	[see 21]	\\
|\citep[see][chap.~2]{jon90}|	&	(see Jones et al., 1990, chap. 2)	&	[see 21, chap. 2]	\\\\
|\citet*{jon90}|	&	Jones, Baker, and Williams (1990)	&		\\
|\citep*{jon90}|	&	(Jones, Baker, and Williams, 1990)	&		\\\\
|\citep{jon90a,jon90b}|	&	(Jones et al., 1990a,b)	&	[21, 32]
\end{tabular}
\end{frame}

\subsection{biblatex}
\begin{frame}{Bibliografien mit biblatex}%{|biblatex|}
\begin{itemize}
\item sämtliche Layouteinstellungen über reine \LaTeX-Makros definiert
\item andere Syntax als die „alten“ Pakete (verwenden \BibTeX-Code)
\item Einstellungen über Paketoptionen\\(siehe \pkg{biblatex}-Dokumentation, Abschn. 3: User Guide)
\item Möglichkeit zur Erstellung mehrerer Bibliografien\\(z.\,B. je Kapitel, oder automatisch nach bestimmten Kriterien)
\end{itemize} 
\end{frame}

\begin{frame}[fragile]{Erstellung der Bibliografie}{|biblatex|}
\begin{block}{im Dokument}
\begin{lstlisting}
\usepackage[style=authoryear]{biblatex}
\addbibresource{bibfile.bib}
\begin{document}
  Text  ... \cite{Tolkien54} ... text.
  \printbibliography
\end{document}
\end{lstlisting}
\end{block}
\begin{block}{in der Konsole}
|$ xelatex dokument.tex|\\
|$ biber dokument.bcf|\\
|$ xelatex dokument.tex|
\end{block}
\end{frame}

\begin{frame}[fragile, shrink]{Zitier- und Bibliografiestile}{|biblatex|}
\begin{itemize}
\item \pkg{biblatex} unterstützt viele vordefinierte Stile:
\item |\usepackage[style=|\meta{Stil}|]{biblatex}|
\end{itemize}
\begin{description}
\item[|numeric|] Standard-Stil \hfill [1, 2, 4, 3, 7]
\item[|numeric-comp|] Kompakte Version von |numeric| \hfill [1-4, 7]
\item[|alphabetic|] Abkürzungen von Autor und Jahr \hfill [Jon95] [JW86] 
\item[|authoryear|] Autor-Jahr-Stil \hfill Jones 1995 
\item[|authoryear-ibid|] Mehrfachnennungen auf einer Seite werden mit \emph{ebd.} abgekürzt
\end{description}
\begin{itemize}
\item Bibliografiestil wird dem Zitierstil angepasst
\item kann mit |citestyle=| und |bibstyle=| verändert werden
\end{itemize}
\end{frame}





\section{Geschicktes Kompilieren}
\subsection{Makefiles}
\begin{frame}[fragile, shrink]{Makefiles}
\begin{itemize}
\item  Ständiges Wechsel des Kompilers (|xelatex|, |biber|, |xelatex|) ist nervig
\pause
\item Mögliche Lösung: Makefiles (Programm |make|)
\end{itemize}
\begin{block}{im Makefile}
\begin{lstlisting}
latex: @xelatex dokument.tex
biber: @biber dokument
clean: @rm -v *.aux *.log *.synctex.gz
       @rm -v *.nav *.snm *.out *.toc
       ...
all: latex biber latex
\end{lstlisting}
\end{block}
\begin{block}{in der Konsole}
|$ make latex|\\
|$ make all|
\end{block}
\end{frame}

\subsection{latexmk}
\begin{frame}[fragile]{|latexmk|}
\begin{itemize}
\item Programm |latexmk| versucht benötigte Programme automatisch aufzurufen
\item kann mittels Datei |latexmkrc| konfiguriert werden
\end{itemize}
\begin{block}{in der Konsole}
|$ latexmk -latex="xelatex --shell-escape %S" \|\\
|  dokument.tex|\\
|$ latexmk -c dokument.tex|
\end{block}
\begin{itemize}
\item genauere Infos: \href{http://man.cx/latexmk}{|man latexmk|} oder \href{http://mirrors.ctan.org/support/latexmk/latexmk.pdf}{|texdoc latexmk|}
\end{itemize}
\end{frame}


\section{Mehrsprachige Dokumente}
\subsection{babel}
\begin{frame}[fragile]{Mehrsprachigkeit mit |babel|}{Sprachen laden}
\pkg{babel} kann mit mehreren Sprachen als Paketoption geladen werden, letzte Sprache ist Hauptsprache:
\begin{lstlisting}
\usepackage[dutch,ngerman]{babel}
\end{lstlisting}
Alternativ: Angabe mit |main=|\meta{Srache}
\begin{lstlisting}
\usepackage[main=ngerman,dutch]{babel}
\end{lstlisting}
\end{frame}


\makeatletter
\def\@xobeysp{ }
\makeatother

\begin{frame}[fragile]{Mehrsprachigkeit mit |babel|}{Sprache umschalten}
Befehl |\selectlanguage{|\meta{Sprache}|}| schaltet Sprache um\\
Umgebung |\begin{otherlanguage}{|\meta{Sprache}|}| nur lokal\\
Befehl |\foreignlanguage{|\meta{Sprache}|}{|\meta{Text}|}| für einzelne Wörter
\vfill\pause
\begin{lstlisting}
% in der Präambel:
\usepackage[main=german,dutch,greek,russian]{babel}

% im Dokument:
Deutscher Text
\selectlanguage{dutch}
Deze tekst is in het Nederlands maar met enkele Duitse \foreingnlanguage{german}{Wörter}

\begin{foreignlanguage}{russian}
, сегодня, является \today
\end{foreignlanguage}
\end{lstlisting}
\end{frame}

\subsection{polyglossia}
\begin{frame}[fragile]{Mehrsprachigkeit mit |polyglossia|}{Sprachen laden}
	|\setmainlanguage[|\meta{Optionen}|]{|\meta{Sprache}|}|
	|\setotherlanguage[|\meta{Optionen}|]{|\meta{Sprache}|}|
	|\setotherlanguages{|\meta{Sprachen}|}|\\[1em]\pause
	\begin{center}
		\scriptsize
		\begin{tabular}{*{5}{l}}
			\multicolumn{5}{l}{\normalsize Vefügbare Sprachen:}\\
			\toprule
			albanian & danish & icelandic & nko & slovenian\\
			amharic & divehi & interlingua & norsk & spanish\\
			arabic & dutch & irish & nynorsk & swedish\\
			armenian & english & italian & occitan & syriac\\
			asturian & esperanto & kannada & piedmontese & tamil\\
			bahasai & estonian & khmer & polish & telugu\\
			bahasam & farsi & korean & portuges & thai\\
			basque & finnish & lao & romanian & tibetan\\
			bengali & french & latin & romansh & turkish\\
			brazil[ian] & friulan & latvian & russian & turkmen\\
			breton & galician & lithuanian & samin & ukrainian\\
			bulgarian & german & lsorbian & sanskrit & urdu\\
			catalan & greek & magyar & scottish & usorbian\\
			coptic & hebrew & malayalam & serbian & vietnamese\\
			croatian & hindi & marathi & slovak & welsh\\
			czech \\
			\bottomrule
		\end{tabular}
	\end{center}
\end{frame}

\begin{frame}[fragile]{Mehrsprachigkeit mit |polyglossia|}{Sprache umschalten}
Befehl |\text|\meta{Sprache}|{|\meta{Text}|}| für einzelne Wörter\\
Umgebung |\begin{|\meta{Sprache}|}| für längere Passagen
\vfill\pause
\begin{lstlisting}
% in der Präambel:
\setmainlanguage{english}
\setotherlanguage{french, greek}

% im Dokument:
The document body is in English, but single words can be in \textgreek{ ελληνικά} or \textfrench{français}.

\begin{french}
  Il est également possible d'écrire des phrases entières en français.
\end{french}
\end{lstlisting}
\end{frame}

\subsection{bidi}
\begin{frame}[fragile]{Unterschiedliche Textrichtungen}
\hologo{XeTeX} unterstützt wechselnde Textrichtungen innerhalb eines Dokuments mit dem Paket \pkg{bibi} (wird von \pkg{polyglossia} automatisch geladen)
\begin{lstlisting}
\usepackage{polyglossia}
\setmainlanguage{german}
\setotherlanguage{arabic}
\begin{document}
  Ein Satz auf Deutsch.
  \begin{Arabic}

  \end{Arabic}
\end{document}
\end{lstlisting}
\end{frame}


\nocite{biblatex, dante:bibliografien, babel, polyglossia, bidi}
\begin{frame}[allowframebreaks]{Weiterführende Literatur}
\printbibliography
\end{frame}

\end{document}
