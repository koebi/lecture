% !TEX TS-program = xelatex
% !TEX encoding = UTF-8 Unicode
% !TEX spellcheck = de_DE
% 
% © 2016 Moritz Brinkmann, CC-by-sa
% http://latexkurs.github.io

\documentclass[
%	vorläufig, 
	blattnr=0,
	ausgabe=2016-10-21,
	abgabe=2016-10-28,
	lösung,
	shortverb,
]{../tex/latexkurs-exercise}

\begin{document}
\begin{abstract}
	\noindent Achtung: Da die Installation der \TeX-Distribution grundlegend für den Kurs ist,
	muss die Abgabe für dieses Blatt von jedem einzeln bearbeitet werden. \\
	\emph{Keine Gruppenabgabe!}
\end{abstract}

\begin{aufgabe}[12]{Minimales \TeX-Dokument}
	Grundlage des \LaTeX-Kurses ist eine funktionsfähige \TeX-Distribution. 
	\begin{enumerate}[label=\alph*)]
		\item Installieren Sie ein lauffähiges \TeX-System\footnote{ Es steht Ihnen frei, \TeX~Live, MiK\TeX, Mac\TeX~oder Pro\TeX t zu installieren, solange die \TeX-Distribution nicht älter als ein Jahr ist. In der Vorlesung wird von einer \href{http://www.tug.org/texlive/}{\TeX~Live 2016}-Installation ausgegangen.} auf Ihrem Rechner. Konsultieren Sie hierzu die \href{http://latexkurs.github.io/exercises/00_texlive_installation.pdf}{Anleitung auf der Vorlesungshomepage}. Machen Sie sich mit dem System vertraut, testen Sie verschiedene Befehle …

		\item \label{aufg:texdoc} Erstellen Sie nun ein minimales plain\TeX-Dokument. Verwenden Sie dazu einen Texteditor und kompilieren Sie über die Kommandozeile! \\(Befehl: |pdftex mydocument.tex|)

		Außer normalem Text soll nur ein einziges \TeX-Kommando verwendet werden (welches und warum genau dieses?). Schreiben Sie |ä,ö,ü,ß| als |ae,oe,ue,ss| und verfassen Sie einen kurzen Text (zwei bis drei Sätze) darüber, welche Themen Sie gerne in der Vorlesung behandeln würden.

		\item \label{aufg:latexdoc} Erstellen Sie weiterhin ein minimales \LaTeX-Dokument, das mindestens „Hallo Welt!“ in eine pdf-Datei ausgibt. Die Wahl der \TeX-Maschine ist dabei Ihnen überlassen.
	\end{enumerate}
	\abgabe{Beide Quelltexte per Mail und das fertige \TeX-Dokument (das PDF von Teil \ref{aufg:texdoc} und \ref{aufg:latexdoc}) als Ausdruck.}
\end{aufgabe}


\lösung{00_loesung_1}


\end{document}