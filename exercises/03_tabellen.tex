% !TEX TS-program = xelatex
% !TEX encoding = UTF-8 Unicode
% !TEX spellcheck = de_DE
% 
% © 2015–2017 Moritz Brinkmann, CC-by-sa
% http://latexkurs.github.io

\documentclass[
	vorläufig=true, 
	blattnr=3,
	ausgabe=2017-11-10,
	abgabe=2017-11-17,
	lösung=false,
	shortverb=false,
]{../tex/latexkurs-exercise}

\usepackage{
	booktabs,
	colortbl,
	hhline,
}



\begin{document}


\begin{aufgabe}[6]{Schöne Tabelle}
	Mithilfe des Pakets \verb|booktabs| ist es möglich, typografisch anspruchsvolle Tabellen zu setzen. Die Dokumentation\footnote{\texttt{texdoc booktabs}} liefert viele Gestaltungshinweise und Best-Practice-Beispiele. Die folgende Tabelle hält sich leider nicht so richtig an diese Empfehlungen.

	\noindent
	\begin{minipage}{\textwidth}
		\centering
		\begin{tabular}{|p{2.1cm}||p{2.5cm}|cl|lr|}
			\hhline{=-----}
			Produkt & Herkunft & Saisonbeginn & Saisonende & Handelsklasse & verfügbar\\\hhline{======}
			Auberginen & Frankreich & Juli & September & I & – \\\hline
			Esskastanien & Frankreich & September & September & I & – \\\hline
			Feldsalat & Deutschland & Oktober & Februar & II & ja \\\hline
			Kürbis & Deutschland & August & Dezember & I & ja\\\hline
			Rote Beete & Italien & September & Februar & I  & ja\\\hline
			Zucchini & Spanien & Juni & Oktober & II &  –\\\hline
			Zwiebeln & Deutschland & Mai & Oktober & – & –\\\hhline{======}
		\end{tabular}
	\end{minipage}
	
	\begin{enumerate}[label=\alph*)]
		\item Korrigieren Sie dieses Manko und setzen Sie die Tabelle nach allen Regeln der Kunst in ein \LaTeX-Dokument. Denken Sie dabei auch über die in der Vorlesung vorgestellten Möglichkeiten des \verb|multirow|-Paktets und des \texttt{\textbackslash multicolumn}-Befehls nach.
		\item Setzen Sie die Tabelle außerdem in eine geeignete Gleitumgebung und versehen Sie sie mit einem vielsagenden Titel.
		\item Geben Sie \emph{handschriftlich} einen kurzen Kommentar zu den von Ihnen vorgenommenen Änderungen an. (Grund für die Änderungen, Probleme, Alternativlösungen, …)
	\end{enumerate}
	\abgabe{Den Quelltext per Mail, Quellcode und fertiges Dokument als Ausdruck. Handschriftlicher Kommentar auf dem Ausdruck.}
\end{aufgabe}

%\newpage
\lösung{03_loesung_schoen}
%\clearpage

\begin{aufgabe}[6]{Bunte Tabelle}
Bisher waren alle Übungsaufgaben in klassischem schwarz/weiß gehalten. An dieser Stelle soll aber gezeigt werden, dass \LaTeX\ sehr wohl auch mit Farben umgehen kann – und zwar am Beispiel einer Tabelle.

	Farben können in \LaTeX\ mit dem Paket \verb|color| verwendet werden, das den Befehl \texttt{\textbackslash color\{\}} zur Verfügung stellt. \texttt{\textbackslash color\{\}} nimmt als Argument eine bekannte Farbe und stellt auf diese um, z.\,B. \texttt{\textbackslash color\{blue\}}. Angaben in rgb-Code o.\,ä. sind auch möglich, z.\,B. \texttt{\textbackslash color[gray]\{0.8\}}. Das Paket \verb|xcolor| erweitert die Möglichkeiten des \texttt{\textbackslash color}-Befehls noch enorm. (Unter anderem ist die Angabe einer Wellenlänge statt Farbe möglich.)

	Studieren Sie die Dokumentation des Paketes \verb|colortbl| und produzieren Sie mithilfe dieses Paketes ein Minimalbeispiel, das eine Tabelle enthalten soll, die wie die folgende aussieht: % der Code ist natürlich keine Vorlage, sondern soll die geforderte Lösung mit anderer Implementierung visualisieren

\begin{center}
	\captionof{table}{Eine wichtige Tabelle hat immer eine vielsagende Beschriftung!}
	\begin{tabular}{l!{\hspace*{-4.3mm}}l}
		\colorbox{gray}{\textcolor{black}{Wochentag}} & \colorbox{gray}{\textcolor{black}{Buchstaben\vphantom{Wg}}}\\
		\colorbox{blue}{\textcolor{white}{Montag\hspace*{.8em}}} & 6 \\
		\colorbox{blue}{\textcolor{white}{Dienstag\hspace*{.34em}}} & 8\\
		\colorbox{blue}{\textcolor{white}{Mittwoch}} &  8\\
		\vdots & \vdots \\
	\end{tabular}
\end{center}

	\abgabe{Den Quelltext per Mail und als Ausdruck.}
\end{aufgabe}

\lösung{03_loesung_bunt}



\end{document}