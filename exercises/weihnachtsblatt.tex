% !TEX TS-program = xelatex
% !TEX encoding = UTF-8 Unicode
% !TEX spellcheck = de_DE
% 
% © 2016 Moritz Brinkmann, CC-by-sa
% http://latexkurs.github.io

\documentclass[
	vorläufig=true, 
	blattnr=W,
	ausgabe=2016-12-16,
	abgabe=2017-01-13,
	lösung,
	shortverb,
]{../tex/latexkurs-exercise}

\chead{\textbf{Weihnachtslatt}}

\begin{document}

\begin{abstract}
	\noindent Die folgenden Aufgaben haben alle mehr oder weniger viel mit der Vorlesung zu tun … Sie sollen 
	Ihnen aber die Gelegenheit bieten, sich – falls Ihnen über die Weihnachtstage langweilig werden sollte – 
	ein wenig mehr mit Ihrem neuen Lieblings\/text\/satz\/system, zu beschäftigen. 
	
	Alle Aufgaben sind Bonusaufgaben – dafür sind sie teilweise recht anspruchsvoll. Punkte werden vor allem 
	für besonders kreative Lösungen und Ansätze vergeben. Sollten Sie über Weihnachten lieber Zeit mit Ihrer 
	Familie oder Ihrem Kater verbringen, verpassen Sie auch nichts, wenn Sie die Aufgaben nicht bearbeiten.
\end{abstract}

\begin{aufgabe}*<4>{\TeX nische Weihnachtsdekoration}
	\begin{enumerate}[label=\alph*)]
		\item Zeigen Sie Ihre Zeichenkünste und erstellen Sie ein – möglichst schönes und \TeX nisch 
		anspruchsvolles – weihnachtliches Motiv mit \TikZ. Ihrer Kreativität sind keine Grenzen gesetzt.
		\item Wenn Sie glauben Ihr Bild könnte es mit denen auf
		\href{http://www.texample.net/tikz/examples/nontech/christmas/}{\TeX ample.net}
		aufnehmen, schicken Sie den Sourcecode ruhig an den Betreiber, Stefan Kottwitz, mit dem Vorschlag, 
		auch Ihr Beispiel zu übernehmen.
		\item Alternativ können Sie versuchen, Ihrem Bild Leben einzuhauchen: Mit dem Paket \pkg{animate} 
		lassen sich sich einfache Animationen aus einzelnen Frames erstellen.\footnote{Leider werden diese 
		Animationen nicht in jedem PDF-Viewer korrekt angezeigt.}
		
		 Falls sich Ihr Motiv dazu eignet, erstellen Sie mehrere Versionen, die sich leicht unterscheiden und 
		 fügen diese später zu einer Animation zusammen.\footnote{Ein PDF mit den genau den Dimensionen des 
		 Inhalts (also ohne Papierformat und ohne Rand) können Sie mit der Dokumentenklasse |standalone| 
		 erstellen.}
	\end{enumerate}
	\abgabe{Quellcode und fertiges Dokument ausgedruckt, Quellcode per Mail.}
\end{aufgabe}



\begin{aufgabe}*<4>{Weihnachts-Rätselheft}
	Ein beliebter Ferienspaß ist das Lösen von Kreuzworträtseln, Sudokus und sonstigen Spielchen. Überraschen 
	Sie Ihre Lieben doch mal mit einem selbst ge\TeX ten Rätselheft. 
	
	Verschaffen Sie sich einen Überblick, welche Pakete es auf \href{http://ctan.org/topic/games}{CTAN} gibt, 	die für den Satz solcher Rätsel geeignet sind und erstellen Sie damit ein kleines Rätselheft, das einem 
	eine Weile die Zeit vertreiben kann.
	\abgabe{Quellcode und fertiges Dokument ausgedruckt, Quellcode per Mail.}
\end{aufgabe}



\begin{aufgabe}*<4>{\TeX-Adventskalender}
	Erstellen Sie einen Adventskalender – also Dokument, das, je nach dem an welchem Datum es kompiliert 
	wird, etwas anderes beinhaltet. Um zu unterscheiden welcher Tag ist können Sie zum Beispiel 
	\hologo{LuaLaTeX} verwenden und mit |\directlua{|\meta{lua-Code}|}| beliebigen lua-Code ausführen, oder 
	das Paket \pkg{ifthen} zu Hilfe nehmen, mit dem man if-Statements in \LaTeX\ erzeugen kann.
	\abgabe{Quellcode ausgedruckt und per Mail.}
\end{aufgabe}

\lösung{w_loesung_adventskalender}


\begin{aufgabe}*<6>{Quine}
	Eine besondere Freude für die Programmiererin und den Programmierer sind Programme, die ihren eigenen 
	Code ausgeben, ohne dabei eine Eingabe zu benötigen. – sogenannte Quines. Probieren Sie mit \TeX\ 
	einen Quine zu erzeugen! Die \TeX-Datei soll also im pdf genau den Code ausgeben, mit dem sie selbst 
	geschrieben ist. Es sollen keine zusätzlichen Befehle im Quellcode stehen, die nicht im pdf als 
	Ausgabe erscheinen!
	
	Da \TeX die Möglichkeit hat, auf seinen eigenen Sourcecode zuzugreifen lässt sich diese Aufgabe genaugenommen relativ einfach lösen. Als richtiger Quine zählt aber nur, was ohne diese Funktion auskommt.
	\abgabe{Quellcode\,/\,fertiges Dokument ausgedruckt, Quellcode per Mail.}
\end{aufgabe}

\lösung{w_loesung_quine}

\end{document}
